\documentclass{scrartcl}
\usepackage{graphicx}  % required for inserting images
\usepackage{float}

% for cyrillic and Bulgarian language
\usepackage[utf8]{inputenc}
\usepackage[T2A]{fontenc}
\usepackage[bulgarian]{babel}

% math packages
\usepackage{amsmath}
\usepackage{float}
\usepackage{amsfonts}
\usepackage{amssymb}

\usepackage{ragged2e}  % adds FlushLeft

\usepackage{listings}
\usepackage{xcolor}

\definecolor{codegreen}{rgb}{0,0.6,0}
\definecolor{codegray}{rgb}{0.5,0.5,0.5}
\definecolor{codepurple}{rgb}{0.58,0,0.82}
\definecolor{backcolour}{rgb}{0.95,0.95,0.92}

\lstdefinestyle{mystyle}{
    language=Octave,
    backgroundcolor=\color{backcolour},   
    commentstyle=\color{codegreen},
    keywordstyle=\color{magenta},
    numberstyle=\tiny\color{codegray},
    stringstyle=\color{codepurple},
    basicstyle=\ttfamily\footnotesize,
    breakatwhitespace=false,         
    breaklines=true,
    captionpos=b,
    keepspaces=false,                 
    %numbers=left,                    
    numbersep=5pt,
    showspaces=false,
    showstringspaces=false,
    showtabs=false,
    tabsize=2,
    basicstyle=\ttfamily\footnotesize,
    columns=fullflexible
}

\lstset{style=mystyle}

\title{Семинар 03}
\author{Ивайло Андреев}
\date{6 март 2025}

\begin{document}
\maketitle  % template title

\section{Уравнения от вид на дробно-линейна функция}

\subsection{Общ случай}

Уравнение от вид на дробно-линейна функция наричаме ДУ от следния вид:

$$y' = f\left(\dfrac{ax+by+c}{mx+ny+p}\right)$$

Нека разгледаме правите:

$$
\begin{cases}
l_1 := ax+by+c = 0 \\
l_2 := mx+ny+p = 0
\end{cases}
$$

Ако $l_1 || l_2$ или $l_1 \equiv l_2$, то даденото уравнение е от вид на линейна функция.

Ако системата е определена, то я решаваме и намираме решението и $(x_0, y_0)$.

Полагаме:

$$u = x - x_0$$

$$v = y - y_0$$

Изразяваме $y'$

$$y(x) = v(u) + y_0$$

$$y'(x) = v'(u)$$

$$v' = f\left(\dfrac{a(u+x_0) + b(v+y_0) + c}{m(u+x_0) + n(v+y_0) + p}\right)$$

$$v' = f\left(\dfrac{au + bv + [ax_0 + by_0 + c]}{mu + nv + [mx_0 + ny_0 + p]}\right)$$

Изразите в скобите са точно решенията на системата и съответно са равни на $0$.

$$v' = f\left(\dfrac{au + bv}{mu + nv}\right)$$

Делим в числител и в знаменател на $u \ne 0$.

$$v' = f\left(\dfrac{a + b\frac{v}{u}}{m + n\frac{v}{u}}\right)$$

Получихме хомогенно уравнение.

\subsection{2023г., контролно 1, вариант x, задача 2}

$$y' = \dfrac{3y-3x-3}{y+3x+3}$$

Решаваме системата

$$
\begin{cases}
3y-3x-3=0\\
y+3x+3 = 0
\end{cases}
$$

Решението на системата е $(x_0, y_0) = (-1, 0)$

Полагаме

$$u = x+1$$

$$v = y$$

$$v' = \dfrac{3v-3u+3-3}{v+3u-3+3} = \dfrac{3v-3u}{v+3u} = \dfrac{3\frac{v}{u}-3}{\frac{v}{u}+3}$$

Получихме хомогенно уравнение

Полагаме

$$z = \dfrac{v}{u}$$

Така

$$v = zu$$

$$v' = z'u + z$$

Заместваме в уравнението

$$z'u + z = \dfrac{3z-3}{z+3}$$

$$z'u = \dfrac{3z-3}{z+3} - z$$

$$z'u = \dfrac{3z-3}{z+3} - \dfrac{z^2+3z}{z+3}$$

$$z'u = \dfrac{-z^3-3}{z+3}$$

$$\dfrac{z+3}{z^2+3}z' = -\dfrac{1}{u}$$

$$\displaystyle \int \dfrac{z+3}{z^2+3} \space dz = -\int \dfrac{1}{u} \space du$$

$$\displaystyle \int \dfrac{z}{z^2+3} \space dz + 3\int \dfrac{1}{z^2+3} \space dz = -\int \dfrac{1}{u} \space du$$

$$\displaystyle \dfrac{1}{2}\int \dfrac{1}{z^2+3} \space d(z^2+3) + \sqrt{3}\int \dfrac{1}{(\frac{z}{\sqrt{3}})^2+1} \space d\frac{z}{\sqrt{3}} = -\int \dfrac{1}{u} \space du$$

$$\dfrac{1}{2}\ln{|z^2+3|}+\sqrt{3}\arctan{\frac{z}{\sqrt{3}}}=-\ln{|u|}+C$$

$$\dfrac{1}{2}\ln{\left|\left(\frac{y}{x+1}\right)^2+3\right|}+\sqrt{3}\arctan{\left(\dfrac{1}{\sqrt{3}}\times\dfrac{y}{x+1}\right)}=-\ln{|x+1|}+C$$

\subsection{2024г., контролно 1, вариант C, задача 1}

$$
\begin{cases}
y' = \left(\dfrac{y+4x-8}{4x-4}\right)^2\\
y(0) = 4
\end{cases}
$$

Решаваме системата

$$
\begin{cases}
y+4x-8=0\\
4x-4 = 0
\end{cases}
$$

Решението на системата е $(x_0, y_0) = (1, 4)$

Полагаме

$$u = x-1$$

$$v = y - 4$$

$$v' = \left(\dfrac{v+4+4u+4-8}{4u+4-4}\right)^2 = \left(\dfrac{v+4u}{4u}\right)^2 = \left(\dfrac{v}{4u}+1\right)^2$$

Получихме хомогенно уравнение

Полагаме

$$z = \dfrac{v}{u}$$

Така

$$v = zu$$

$$v' = z'u + z$$

Заместваме в уравнението

$$z'u + z = \dfrac{1}{16}(z+4)^2$$

$$16z'u + 16z = z^2+8z+16$$

$$16z'u = z^2-8z+16$$

$$16z'u = (z-4)^2$$

$$\dfrac{z'}{(z-4)^2} = \dfrac{1}{16u}$$

$$\displaystyle \int\dfrac{1}{(z-4)^2}\space d(z-4) = \dfrac{1}{16}\int\dfrac{1}{u}\space du$$

$$-\dfrac{1}{z-4} = \dfrac{\ln{|u|}}{16} + C$$

$$z = 4 - \dfrac{16}{16C+\ln{|u|}}$$

$$\dfrac{y-4}{x-1} = 4-\dfrac{16}{16C+\ln{|x-1|}}$$

Прилагаме началното условие $y(0) = 4$

$$\dfrac{4-4}{0-1} = 4 - \dfrac{16}{\ln{|-1|} + 16C}$$

$$0 = 4-\dfrac{16}{16C}$$

$$C = \dfrac{1}{4}$$

Така получаваме окончателно решение

$$y = 4 + (x-1)\left(4-\dfrac{16}{4+\ln{|x-1|}}\right)$$

\section{Линейни уравнения от първи ред}

Линейно уравнение от първи ред наричаме ДУ от следния вид:

$$y'(x) = a(x)y(x) + b(x)$$

където $a(x), b(x)$ са непрекъснати функции в затворен интервал.

Общото решение се дава със следната формула:

$$y(x) = \displaystyle e^{\int a(x)\space dx}\left( C + \int b(x) e^{-\int a(x) \space dx} \space dx \right)$$

\subsection{Задача 1.5.3 от учебника на проф. Огнян Христов}

$$y' = -y\sin{x} + \mathrm{e}^{\cos{x}}$$

$$a(x) = -\sin{x}$$

$$b(x) = \mathrm{e}^{\cos{x}}$$

Формулата изисква да сметнем интегралите $I = \int a(x) \space dx$ и $-I$

$$I = \int -\sin{x} \space dx = \cos{x} + C_1$$

$$-I = -\cos{x} - C_1$$

Прилагаме формулата

$$y = \mathrm{e}^I\left(C + \int \mathrm{e}^{\cos{x}}\mathrm{e}^{-I}\space dx\right)$$

$$y = \mathrm{e}^{\cos{x}+C_!}\left(C + \int \mathrm{e}^{\cos{x}}\mathrm{e}^{-\cos{x}-C_1}\space dx\right)$$

$$y = \mathrm{e}^{C_1}\mathrm{e}^{\cos{x}}\left(C + \mathrm{e}^{-C_1}\int \mathrm{e}^{\cos{x}}\mathrm{e}^{-\cos{x}}\space dx\right)$$

$$y = \mathrm{e}^{C_1}\mathrm{e}^{\cos{x}}\left(C + \mathrm{e}^{-C_1}\int \mathrm{e}^{\cos{x}-\cos{x}}\space dx\right)$$

$$y = \mathrm{e}^{C_1}\mathrm{e}^{\cos{x}}\left(C + \mathrm{e}^{-C_1}\int dx\right)$$

Също виждаме, че втората константа може да бъде елиминирана лесно. Обикновено дори не пишем произволната константа при примитивните в такъв тип задачи освен ако няма конкретна причина - например ако искаме константата да разкрие модул.

$$y = \mathrm{e}^{\cos{x}}\left(\mathrm{e}^{C_1}C + \mathrm{e}^{C_1}\mathrm{e}^{-C_1}\int dx\right)$$

$$y = \mathrm{e}^{\cos{x}}\left(C_2 + \int dx\right)$$

$$y = \mathrm{e}^{\cos{x}}\left(C_2 + x + C_3 \right)$$

$$y = \mathrm{e}^{\cos{x}}\left(C_4 + x\right)$$

\subsection{2024г., изпит-задачи, вариант D, задача 1}

$$x(x+2)^2y' = 4(x+2)y + x(x+1)$$

$$y' = \dfrac{4}{x(x+2)}y+\dfrac{x+1}{(x+2)^2}$$

$$a(x) = \dfrac{4}{x(x+2)}$$

$$b(x) = \dfrac{x+1}{(x+2)^2}$$

Формулата изисква да сметнем интегралите $I = \int a(x) \space dx$ и $-I$

\begin{align*}
\displaystyle I
&= \int a(x) \space dx\\
&= \int\dfrac{4}{x(x+2)} \space dx\\
&= 4\int\left(\dfrac{A}{x} + \dfrac{B}{x+2}\right)\space dx\\
\end{align*}

$$\dfrac{A}{x} + \dfrac{B}{x+2} = \dfrac{1}{x(x+2)} = \dfrac{Ax+2A+Bx}{x(x+2)}$$

$$(B+A)x+2A=0x+1$$

Така

$$B+A=0;\quad 2A=1\implies A=\dfrac{1}{2};\space B=-\dfrac{1}{2}$$

\begin{align*}
\displaystyle I
&= 4\int\left(\dfrac{A}{x} + \dfrac{B}{x+2}\right)\space dx\\
&= 4\int\left(\dfrac{1}{2x} - \dfrac{1}{2(x+2)}\right)\space dx\\
&= 2\int\dfrac{1}{x}\space dx - 2\int\dfrac{d(x+2)}{x+2}\\
&= 2\ln{|x|} - 2\ln{|x+2|}\\
&= 2(\ln{|x|} - \ln{|x+2|})\\
&= 2\ln{\left|\dfrac{x}{x+2}\right|}\\
&= \ln{\left|\dfrac{x}{x+2}\right|^2}\\
&= \ln{\left(\dfrac{x}{x+2}\right)^2}
\end{align*}

\begin{align*}
\displaystyle -I
&= -\ln{\left(\dfrac{x}{x+2}\right)^2}\\
&= \ln{\left(\dfrac{x}{x+2}\right)^{-2}}\\
&= \ln{\left(\dfrac{x+2}{x}\right)^{2}}\\
\end{align*}

$$y = \mathrm{e}^{\ln{\left(\dfrac{x}{x+2}\right)^2}}\left(C+\int\dfrac{x+1}{(x+2)^2}\mathrm{e}^{\ln{\left(\dfrac{x+2}{x}\right)^2}}\space dx\right)$$

$$y = \left(\dfrac{x}{x+2}\right)^2\left(C+\int\dfrac{x+1}{(x+2)^2}{\dfrac{(x+2)^2}{x^2}}\space dx\right)$$

$$y = \left(\dfrac{x}{x+2}\right)^2\left(C+\int\dfrac{x+1}{1}{\dfrac{1}{x^2}}\space dx\right)$$

$$y = \left(\dfrac{x}{x+2}\right)^2\left(C+\int\dfrac{1}{x}\space dx + \int\dfrac{1}{x^2}\space dx\right)$$

$$y = \left(\dfrac{x}{x+2}\right)^2\left(C+\ln{|x|}-\dfrac{1}{x}\right)$$

\section{Уравнение на Бернули}

\subsection{Общ случай}

Уравнение на Бернули наричаме ДУ от следния вид:

$$y'(x) = a(x)y(x) + b(x)y^n$$

където $a(x), b(x)$ са непрекъснати функции в затворен интервал и $n \in \mathbb{R}$.

Ако $n = 0$ получаваме линейно уравнение.

Ако $n = 1$ получаваме уравнение с разделящи се променливи.

Делим на $y^n$ при $y \ne 0$.

$$\dfrac{y'}{y^n} = a(x)y^{1-n}+b(x)$$

$$\text{Полагаме } z = y^{1-n}$$

$$z' = (1-n)y^{-n}y'$$

$$y' = \dfrac{z'y^n}{1-n}$$

Заместваме в даденото уравнение.

$$\dfrac{z'y^n}{(1-n)y^n} = a(x)z+b(x)$$

$$z' = (1-n)a(x)z+(1-n)b(x)$$

Получаваме линейно уравнение.

\subsection{2024г., контролно 1, вариант F, задача 1}

$$
\begin{cases}
3(1+x)(1-x)^2y' = 4(x-1)y - 3(x+1)^3\mathrm{e}^{-2x}y^4\\
y(0) = \sqrt[3]{2}
\end{cases}
$$

$$y' = \dfrac{4}{3}\dfrac{1}{(x+1)(x-1)}y-\dfrac{(x+1)^2}{(x-1)^2}\mathrm{e}^{-2x}y^4$$

Вижда се, че $y\equiv 0$ е решение на ДУ, но не на задачата на Коши.

Делим на $y^4\ne 0$

$$\dfrac{y'}{y^4} = \dfrac{4}{3}\dfrac{1}{(x+1)(x-1)}y^{-3}-\dfrac{(x+1)^2}{(x-1)^2}\mathrm{e}^{-2x}$$

Полагаме $z(x) = y^{-3} = \dfrac{1}{y^3}$

Тогава $z'(x) = -3y^{-4}(x)y'(x) \implies y' = -\dfrac{z'y^4}{3}$

$$-\dfrac{z'y^4}{3}\dfrac{1}{y^4} = \dfrac{4}{3}\dfrac{1}{(x+1)(x-1)}z-\dfrac{(x+1)^2}{(x-1)^2}\mathrm{e}^{-2x}$$

$$z' = -4\dfrac{1}{(x+1)(x-1)}z+3\dfrac{(x+1)^2}{(x-1)^2}\mathrm{e}^{-2x}$$

$$a(x) = -4\dfrac{1}{(x+1)(x-1)}; \quad b(x) = 3\dfrac{(x+1)^2}{(x-1)^2}\mathrm{e}^{-2x}$$

$$\displaystyle I = \int a(x)\space dx = -4\int \dfrac{1}{(x+1)(x-1)}\space dx = -4\int\left(\dfrac{A}{x+1}+\dfrac{B}{x-1}\right)\space dx$$

$$\dfrac{1}{(x+1)(x-1)} = \dfrac{A}{x+1}+\dfrac{B}{x-1} = \dfrac{Ax-A+Bx+B}{(x+1)(x-1)}$$

$$x(A+B)+(B-A)=0x+1$$

$$B-A=1;\quad A+B=0$$

$$A=-\dfrac{1}{2};\quad B=\dfrac{1}{2}$$

$$\displaystyle I -4\left(\dfrac{1}{2}\int\dfrac{1}{x-1}\space d(x-1)-\dfrac{1}{2}\int\dfrac{1}{x+1}\space d(x+1)\right)$$

$$I = 2(\ln{|x+1|}-\ln{|x-1|})$$

$$I = 2 \ln{\dfrac{x+1}{x-1}} = \ln{\left(\dfrac{x+1}{x-1}\right)^2}$$

$$-I = -\ln{\left(\dfrac{x+1}{x-1}\right)^2} = \ln{\left(\dfrac{x+1}{x-1}\right)^{-2}} = \ln{\left(\dfrac{x-1}{x+1}\right)^2}$$

$$z=\dfrac{(x+1)^2}{(x-1)^2}\left(C+3\int\dfrac{(x-1)^2}{(x+1)^2}\dfrac{(x+1)^2}{(x-1)^2}\mathrm{e}^{-2x}\space dx\right)$$

$$z=\dfrac{(x+1)^2}{(x-1)^2}\left(C-\dfrac{3}{2}\int\mathrm{e}^{-2x}\space d(-2z)\right)$$

$$z=\dfrac{(x+1)^2}{(x-1)^2}\left(C-\dfrac{3}{2}\mathrm{e}^{-2x}\right)$$

Прилагаме началното условие

$$\dfrac{1}{2}=1\left(C-\dfrac{3}{2}\right)$$

$$C=2$$

Така решението на задачата на Коши е:

$$\dfrac{1}{y^3} = \left(\dfrac{x+1}{x-1}\right)^2\left(2-\dfrac{\mathrm{e}^{-2x}}{2}\right)$$

\subsection{2022г., контролно 1, вариант 2, задача 1}

$$
\begin{cases}
-5x\dfrac{y'}{y^6} = \dfrac{2}{y^5} + x^3\mathrm{e}^x\cos{x} \\
y\left(\dfrac{3\pi}{4}\right) = 1
\end{cases}
$$

Полагаме $z(x) = y^{-5} = \dfrac{1}{y^5}$

Тогава $z'(x) = -5y^{-6}(x)y'(x) \implies y' = -\dfrac{z'y^6}{5}$

$$-5x\dfrac{1}{y^6}y^6\dfrac{1}{-5}z'=2z+x^3\mathrm{e}^x\cos{x}$$

$$z'=\dfrac{2}{x}z+x^2\mathrm{e}^x\cos{x}$$

$$a(x) = \dfrac{2}{x};\quad b(x) = x^2\mathrm{e}^x\cos{x}$$

$$\displaystyle I=\int a(x)\space dx = 2\int\dfrac{1}{x}\space dx = 2\ln{|x|} = \ln{x^2}$$

$$-I = -\ln{x^2} = \ln{x^{-2}}$$

$$z = x^2\left(C+\int\mathrm{e}^x\cos{x}\space dx\right)$$

$$J = \displaystyle\int \mathrm{e}^x\cos{x} \space dx$$

$$J = \displaystyle\int \mathrm{e}^x \space d\sin{x}$$

$$J = \mathrm{e}^{x}\sin{x} - \displaystyle\int \sin{x} \space d\mathrm{e}^x$$

$$J = \mathrm{e}^{x}\sin{x} - \displaystyle\int \sin{x} \space d\mathrm{e}^x$$

$$J = \mathrm{e}^{x}\sin{x} + \displaystyle\int \mathrm{e}^x \space d\cos{x}$$

$$J = \mathrm{e}^{x}\sin{x} + \mathrm{e}^{x}\cos{x} - \displaystyle\int \mathrm{e}^x\cos{x} \space dx$$

$$J = \mathrm{e}^{x}\sin{x} + \mathrm{e}^{x}\cos{x} - J + C^*$$

$$J = \dfrac{\mathrm{e}^x(\sin{x} + \cos{x})}{2} + C$$

$$z = x^2\left(C+\dfrac{\mathrm{e}^x(\sin{x} + \cos{x})}{2}\right)$$

$$\dfrac{1}{y^5} = x^2\left(C+\dfrac{\mathrm{e}^x(\sin{x} + \cos{x})}{2}\right)$$

Прилагаме началното условие $y\left(\dfrac{3\pi}{4}\right) = 1$

$$1=\dfrac{9}{16}\pi^2C \implies C=\dfrac{16}{9\pi^2}$$

$$\dfrac{1}{y^5} = x^2\left(\dfrac{16}{9\pi^2}+\dfrac{\mathrm{e}^x}{2}(\sin{x}+\cos{x})\right)$$

\end{document}
