\documentclass{scrartcl}
\usepackage{graphicx}  % required for inserting images
\usepackage{float}

% for cyrillic and Bulgarian language
\usepackage[utf8]{inputenc}
\usepackage[T2A]{fontenc}
\usepackage[bulgarian]{babel}

% math packages
\usepackage{amsmath}
\usepackage{float}
\usepackage{amsfonts}
\usepackage{amssymb}

\usepackage{ragged2e}  % adds FlushLeft

\usepackage{listings}
\usepackage{xcolor}

\definecolor{codegreen}{rgb}{0,0.6,0}
\definecolor{codegray}{rgb}{0.5,0.5,0.5}
\definecolor{codepurple}{rgb}{0.58,0,0.82}
\definecolor{backcolour}{rgb}{0.95,0.95,0.92}

\lstdefinestyle{mystyle}{
    language=Octave,
    backgroundcolor=\color{backcolour},   
    commentstyle=\color{codegreen},
    keywordstyle=\color{magenta},
    numberstyle=\tiny\color{codegray},
    stringstyle=\color{codepurple},
    basicstyle=\ttfamily\footnotesize,
    breakatwhitespace=false,         
    breaklines=true,
    captionpos=b,
    keepspaces=false,                 
    %numbers=left,                    
    numbersep=5pt,
    showspaces=false,
    showstringspaces=false,
    showtabs=false,
    tabsize=2,
    basicstyle=\ttfamily\footnotesize,
    columns=fullflexible
}

\lstset{style=mystyle}

\title{Семинар 04}
\author{Ивайло Андреев}
\date{13 март 2025}

\begin{document}
\maketitle  % template title

\section{Уравнения, нерешени относно производната}

\subsection{Общ случай}

$$y(x) = f(x, y'(x))$$

Полагаме $z(x) = y'(x)$

Получаваме $y(x) = f(x, z(x))$

Диференцираме по $x$.

Трябва да се получи нещо хубаво.

ВАЖНО! В крайния отговор трябва да има точно 1 константа, тъй като това е ОДУ от първи ред!

\subsection{2024г., контролно 1, вариант C, задача 2}

$$y = x^2(y')^2+x\ln{x}y'+1$$

Полагаме $z(x)=y'(x)$

$$y=x^2z^2+x\ln{x}z+1\qquad (*)$$

Диференцираме у-нието $(*)$ по $x$

$$y'=2xz^2+2x^2zz'+(x\ln{x})'z+x\ln{x}z'$$

$$z=2xz^2+2x^2zz'+(\ln{x}+1)z+x\ln{x}z'$$

$$\not{z}=2xz^2+2x^2zz'+\ln{x}z++\not{z}+x\ln{x}z'$$

$$0=2xz^2+2x^2zz'+\ln{x}z+x\ln{x}z'$$

$$0=2xz(z+xz')+\ln{x}(z+xz')$$

$$0=(z+xz')(2xz+\ln{x})$$

\textbf{Първи случай:} $z+xz'=0$

$$z'x=-z$$

Проверяваме дали $z\equiv 0$ е решение, като заместим в $(*)$. $y=1$ е решение.

Делим на $z\ne 0$

$$\dfrac{z'}{z} = -\dfrac{1}{x}$$

Интегрираме по $x$

$$\displaystyle \int\dfrac{1}{z}\space dz = -\int\dfrac{1}{x}\space dx$$

$$\ln{|z|}=\ln{|x|}+C$$

$$\mathrm{e}^{\ln{|z|}}=\mathrm{e}^{\ln{|\frac{1}{x}|}+C}$$

$$|z|=\mathrm{e}^C\left|\dfrac{1}{x}\right|$$

$$z=\pm C_1\dfrac{1}{x}$$

$$z=C_2\dfrac{1}{x}$$

Заместваме в $(*)$ с $z$

$$y = C_2^2+C_2\ln{x}+1$$

\textbf{Втори случай:} $2xz+\ln{x}=0$

$$z=-\dfrac{\ln{x}}{2x}$$

Заместваме в $(*)$ с $z$

$$y = x^2\dfrac{\ln^2{x}}{4x^2}-x\ln{x}\dfrac{\ln{x}}{2x}+1$$

$$y=\dfrac{\ln^2{x}}{4}-\dfrac{\ln^2{x}}{2}+1$$

$$y=-\dfrac{\ln^2{x}}{4}+1$$

\section{Автономни нелинейни уравнения}

\subsection{Общ случай}

$$f(y, y', y'') = 0$$

$x$ не участва явно.

Възможно е $x$ да не участва явно, но уравнението да не се решава с долуописания алгоритъм.

Полагаме $y'(x) = P(y(x))$

$$y''(x) = P'P$$

Заместваме в уравнението и трябва да се получи нещо хубаво.

Интегрира се по $y$. (Това е единственият тип задача, където се интегрира по $y$)

\subsection{2024г., контролно 1, вариант F, задача 2}

$$y'' = 2\mathrm{e}^y y'(y'-3)^2$$

Полагаме $y'(x) = P(y) = P$

Тогава $y''(x) = (P(y))' = P'(y)y'(x) = P'P$

$$P'P = 2\mathrm{e}^yP(P-3)^2$$

$$P'P - 2\mathrm{e}^yP(P-3)^2 = 0$$

$$P(P' - 2\mathrm{e}^y(P-3)^2) = 0$$

Ако $P\equiv 0$, то $y' \equiv 0\implies y = C_1^*$ е решение. Иначе делим на $P\ne 0$

$$P' - 2\mathrm{e}^y(P-3)^2 = 0$$

$$P' = 2\mathrm{e}^y(P-3)^2$$

Получаваме уравнение с разделящи се променлива за $y$ и $P(y)$

Ако $(P-3)^2\equiv 0$, то $P=3$ и $y' \equiv 3\implies y = 3x+C_2^*$ е решение. Иначе делим на $(P-3)^2\ne 0$

$$\dfrac{P'}{(P-3)^2} = 2\mathrm{e}^y$$

Интеграраме по $y$

$$\displaystyle \int \dfrac{P'}{(P-3)^2}\space dy = \int 2\mathrm{e}^y \space dy$$

$$\displaystyle \int \dfrac{1}{(P-3)^2}\space d(P-3) = 2 \int \mathrm{e}^y \space dy$$

$$-\dfrac{1}{P-3} = 2\mathrm{e}^y - C_1$$

$$\dfrac{1}{P-3} = -2\mathrm{e}^y + C_1$$

$$\dfrac{1}{C_1-2\mathrm{e}^y} = P-3$$

$$P = 3 + \dfrac{1}{C_1-2\mathrm{e}^y}$$

$$y' = 3 + \dfrac{1}{C_1-2\mathrm{e}^y}$$

Получаваме уравнение с разделящи се променлива за $x$ и $y(x)$

$$y' = \dfrac{3C_!-6\mathrm{e}^y+1}{C_!-2\mathrm{e}^y}$$

Нека $C_2 = 3C_!+1 \implies C_! = \dfrac{C_2-1}{3}$

$$y' = \dfrac{3C_2-18\mathrm{e}^y}{C_2-1-6\mathrm{e}^y}$$

$$y'\dfrac{C_2-1-6\mathrm{e}^y}{3C_2-18\mathrm{e}^y} = 1$$

$$y'\dfrac{3C_2-18\mathrm{e}^y-3}{3C_2-18\mathrm{e}^y} = 3$$

$$y'\left(1-3\dfrac{1}{3C_2-18\mathrm{e}^y}\right) = 3$$

$$y'\left(1-\dfrac{1}{C_2-6\mathrm{e}^y}\right) = 3$$

$$y'\left(1+\dfrac{1}{6\mathrm{e}^y-C_2}\right) = 3$$

Интегрираме по $x$

$$\displaystyle \int y'\left(1+\dfrac{1}{6\mathrm{e}^y-C_2}\right)\space dx = \int 3\space dx$$

$$\displaystyle \int \left(1+\dfrac{1}{6\mathrm{e}^y-C_2}\right)\space dy = 3\int dx$$

$$\displaystyle \int dx + \dfrac{1}{6}\int \dfrac{1}{\mathrm{e}^y-\frac{C_2}{6}}\space dy = 3\int dx$$

Решаваме интеграла

$$I = \displaystyle \int \dfrac{1}{\mathrm{e}^y-\frac{C_2}{6}}\space dy \space dx$$

Полага се \( u = e^y \), откъдето \( du = e^y dy = u dy \), следователно

$$I = \int \frac{du}{u(u - \frac{C_2}{6})}$$

Разлагаме на елементарни дроби

$$\frac{1}{u(u - \frac{C_2}{6})} = \frac{A}{u} + \frac{B}{u - \frac{C_2}{6}} = \frac{A (u - \frac{C_2}{6}) + B u}{u(u - \frac{C_2}{6})}$$

$$1 = A (u - \frac{C_2}{6}) + B u$$

$$A = -\dfrac{6}{C_2};\quad B=\dfrac{6}{C_2}$$

$$I = \int \left( -\frac{6}{C_2} \frac{1}{u} + \frac{6}{C_2} \frac{1}{u - \frac{C_2}{6}} \right) du$$

$$I = -\frac{6}{C_2} \ln |u| + \frac{6}{C_2} \ln \left| u - \frac{C_2}{6} \right| + C$$

$$I = \frac{6}{C_2} \ln \left| \frac{e^y - \frac{C_2}{6}}{e^y} \right| + C$$

$$y + \dfrac{\ln{\left|1-\dfrac{C_2}{6\mathrm{e}^y}\right|}}{C_2} = 3x+C_3$$

\section{Уравнения, където липсват първите младши производни}

\subsection{Общ случай}

$$f(x, y', y'') = 0$$

В случая липсват първите 1 най-младши производни, а именно $y$.

Тогава полагаме $p(x) = y'(x)$

\subsection{Задача 2.2.3 от учебника на проф. Огнян Христов}

$$y'' + (y')^2 = y'\left(x+\dfrac{1}{x}\right)$$

Полагаме $p(x) = y'(x)$

$$p' + p^2 = p\left(x+\dfrac{1}{x}\right)$$

$$p' = p\left(x+\dfrac{1}{x}\right) - p^2$$

Получихме уравнение на Бернули

$$z = p^{-1} \implies z' = -p^{-2}p' \implies p' = -z'p^{-2}$$

$$z' = z\left(-x-\dfrac{1}{x}\right) + 1$$

$$I = - \int \left(-x-\dfrac{1}{x}\right)\space dx$$

$$I = -\dfrac{x^2}{2}-\ln{|x|} + C_!$$

$$z = \mathrm{e}^{-\frac{x^2}{2}-\ln{|x|}+C_1}\left(C+\int \mathrm{e}^{+\frac{x^2}{2}+\ln{|x|}-C_1}\space dx\right)$$

$$z = \mathrm{e}^{C_1}\left|\dfrac{1}{x}\right|\mathrm{e}^{-\frac{x^2}{2}}\left(C+\int \mathrm{e}^{C_1}|x|\mathrm{e}^{\frac{x^2}{2}}\space dx\right)$$

$$z = \pm\mathrm{e}^{C_1}\dfrac{1}{x}\mathrm{e}^{-\frac{x^2}{2}}\left(C+\int \dfrac{1}{\pm\mathrm{e}^{C_1}}x\mathrm{e}^{\frac{x^2}{2}}\space dx\right)$$

$$z = C_2\dfrac{1}{x}\mathrm{e}^{-\frac{x^2}{2}}\left(C+\dfrac{1}{C_2}\int x\mathrm{e}^{\frac{x^2}{2}}\space dx\right)$$

$$z = \dfrac{1}{x}\mathrm{e}^{-\frac{x^2}{2}}\left(C_2C+\dfrac{C_2}{C_2}\int x\mathrm{e}^{\frac{x^2}{2}}\space dx\right)$$

$$z = \dfrac{1}{x}\mathrm{e}^{-\frac{x^2}{2}}\left(C_3+\int \mathrm{e}^{\frac{x^2}{2}}\space d\frac{x^2}{2}\right)$$

$$z = \dfrac{1}{x}\mathrm{e}^{-\frac{x^2}{2}}\left(C_3+ \mathrm{e}^{\frac{x^2}{2}}\right)$$

$$\dfrac{1}{p} = \dfrac{C_3}{x}\mathrm{e}^{-\frac{x^2}{2}}+ \dfrac{1}{x}$$

$$\dfrac{1}{y'} = \dfrac{1+C_3}{x}\mathrm{e}^{-\frac{x^2}{2}}$$

$$y' = \dfrac{x}{1+C_3\mathrm{e}^{-\frac{x^2}{2}}}$$

Получаваме уравнение с разделени променливи

$$\displaystyle \int y' \space dx = \int\dfrac{x}{1+C_3\mathrm{e}^{-\frac{x^2}{2}}}\space dx$$

Решаваме интеграла

$$J = \int\dfrac{x}{1+C_3\mathrm{e}^{-\frac{x^2}{2}}}\space dx$$

$$J = -\int\dfrac{1}{1+C_3\mathrm{e}^{-\frac{x^2}{2}}}\space d\left(-\frac{x^2}{2}\right)$$

$$J = -\int\dfrac{1}{1+C_3\mathrm{e}^{u}}\space du$$

$$J = -\int\dfrac{1}{v(1+C_3v)}\space dv$$

$$J = -\dfrac{1}{C_3}\int\dfrac{1}{v(1+\frac{v}{C_3})}\space dv$$

$$\dfrac{1}{v(1+\frac{v}{C_3})} = \dfrac{A}{v} + \dfrac{B}{1+\frac{v}{C_3}} = \dfrac{Av+\frac{A}{C_3}+Bv}{v(1+\frac{v}{C_3})}$$

$$A = C_3; \quad B=-C_3$$

$$J = -\dfrac{1}{C_3}\int\left(\dfrac{C_3}{v}-\dfrac{C_3}{1+\frac{v}{C_3}}\right)\space dx$$

$$J = -\int\left(\dfrac{1}{v}-\dfrac{1}{1+\frac{v}{C_3}}\right)\space dx$$

$$J = -\ln{|v|}+\ln{|1+\frac{v}{C_3}|} + C_4$$

$$J = -\ln{|\mathrm{e}^{-\frac{x^2}{2}}|}+\ln{|1+\frac{\mathrm{e}^{-\frac{x^2}{2}}}{C_3}|} + C_4$$

$$J = \frac{x^2}{2}+\ln{|1+\frac{\mathrm{e}^{-\frac{x^2}{2}}}{C_3}|} + C_4$$

Така решението е

$$y = \frac{x^2}{2}+\ln{|1+\frac{\mathrm{e}^{-\frac{x^2}{2}}}{C_3}|} + C_4$$

\section{Хомогенни уравнения от по-висок ред}

\subsection{2024г., контролно 1, вариант A, задача 2}

$$[3x\sin(3x)-\cos(3x)]yy' = x\cos(3x) [(y')^2-yy'']$$

Полагаме $z = \dfrac{y'}{y}$

$$y' = yz$$

$$y'' = y'z+yz'=yz^2+yz'=y(z^2+z')$$

Заместваме в уравнението

$$[3x\sin(3x)-\cos(3x)]y^2z = x\cos(3x) [y^2z^2-y^2(z^2+z')]$$

$$[3x\sin(3x)-\cos(3x)]y^2z = x\cos(3x) [y^2z^2-y^2z^2-y^2z')]$$

$$[3x\sin(3x)-\cos(3x)]y^2z = -x\cos(3x) y^2z'$$

Получихме уравнение с разделени променливи.

$y\equiv 0$ е решение на уравнението. Делим на $y^2\ne 0$

$$[3x\sin(3x)-\cos(3x)]z = -x\cos(3x) z'$$

$$z' = -\dfrac{3x\sin(3x)-\cos(3x)}{x\cos{(3x)}}z$$

$z\equiv 0$ е решение и тогава $y' = 0$ е решение и съответно $y = C$ е решение. Делим на $z\ne 0$

$$\dfrac{z'}{z} = \dfrac{\cos(3x)-3x\sin(3x)}{x\cos{(3x)}}$$

$$\dfrac{z'}{z} = \dfrac{1}{x}-3\tan{(3x)}$$

$$\int \dfrac{z'}{z}\space dx = \int \dfrac{1}{x}\space dx-\int 3\tan{(3x)}\space dx$$

$$\int \dfrac{1}{z}\space dz = \int \dfrac{1}{x}\space dx-\int \dfrac{\sin{(3x)}}{\cos{(3x)}}\space d(3x)$$

$$\int \dfrac{1}{z}\space dz = \int \dfrac{1}{x}\space dx+\int \dfrac{1}{\cos{(3x)}}\space d\cos{(3x)}$$

$$\ln{|z|} = \ln{|x|}+\ln{|\cos{(3x)}|}+C_1$$

$$\ln{|z|} = \ln{|x|}+\ln{|\cos{(3x)}|}+\ln{C_2}$$

$$\ln{|z|} = \ln{|x\cos{(3x)}C_2|}$$

$$z = \pm C_2 x\cos{(3x)}$$

$$z = C_3 x\cos{(3x)}$$

$$\dfrac{y'}{y} = C_3 x\cos{(3x)}$$

Получихме уравнение с разделени променливи.

$$\int \dfrac{y'}{y}\space dx = \int C_3 x\cos{(3x)}\space dx$$

$$\int \dfrac{1}{y}\space dy =  C_3 \int x\cos{(3x)}\space dx$$

$$I = \int x\cos{(3x)}\space dx$$

$$I = \dfrac{1}{9}\int 3x\cos{(3x)}\space d(3x)$$

$$I = \dfrac{1}{9}\int u\cos{u}\space du$$

$$I = \dfrac{1}{9}\int u\space d\sin{u}$$

$$I = \dfrac{1}{9}\left(u\sin{u} - \int\sin{u}\space du\right)$$

$$I = \dfrac{1}{9}\left(u\sin{u} + \cos{u}\right) + C_4$$

$$I = \dfrac{x\sin{(3x)}}{3} + \dfrac{\cos{(3x)}}{9} + C_4$$

$$\ln{|y|} = \dfrac{C_3x\sin{(3x)}}{3} + \dfrac{C_3\cos{(3x)}}{9} + C_3C_4$$

$$y = \pm \mathrm{e}^{\frac{C_3x\sin{(3x)}}{3} + \frac{C_3\cos{(3x)}}{9} + C_3C_4}$$

\subsection{2022г., контролно 1, вариант 2, задача 2}

$$4xyy'' - x^2y^2 = 4x(y')^2 + 8yy',\quad x>0$$

Полагаме $z = \dfrac{y'}{y}$

$$y' = yz$$

$$y'' = y'z+yz'=yz^2+yz'=y(z^2+z')$$

Заместваме в уравнението

$$4xy^2(z'+z^2) - x^2y^2 = 4xy^2z^2 + 8y^2z$$

$y\equiv 0$ е решение на уравнението. Делим на $y^2\ne 0$

$$4x(z'+z^2) - x^2 = 4xz^2 + 8z$$

$$4xz' + 4xz^2 - x^2 = 4xz^2 + 8z$$

$$4xz' - x^2 = 8z$$

$$4xz' = x^2 + 8z$$

$$z' = \dfrac{x}{4} + z\dfrac{2}{x}$$

$$z' = z\dfrac{2}{x}+\dfrac{x}{4}$$

Получихме линейно уравнение

$$I = \int \dfrac{2}{x}\space dx$$

$$I = 2\int \dfrac{1}{x}\space dx$$

$$I = 2\ln{|x|}$$

$$I = \ln{x^2}$$

$$-I = -\ln{x^2} = \ln{x^{-2}}$$

$$z = \mathrm{e}^{\ln{x^2}}\left(C+\dfrac{1}{4}\int x\mathrm{e}^{\ln{x^{-2}}}\space dx\right)$$

$$z = x^2\left(C+\dfrac{1}{4}\int \dfrac{1}{x}\space dx\right)$$

$$z = x^2\left(C + \frac{1}{4}\ln{|x|}\right)$$

$$z = Cx^2+\dfrac{x^2\ln{|x|}}{4}$$

По условие имаме, че $x>0$ и съответно можем да разкрием модула

$$z = Cx^2+\dfrac{x^2\ln{x}}{4}$$

$$\dfrac{y'}{y} = Cx^2+\dfrac{x^2\ln{x}}{4}$$

Получаваме уравнение с разделени променливи

$$\int \dfrac{y'}{y}\space dx = \int \left(Cx^2+\dfrac{x^2\ln{x}}{4}\right)\space dx$$

$$\int \dfrac{1}{y}\space dy = C\int x^2\space dx+\dfrac{1}{4}\int x^2\ln{x}\space dx$$

Решаваме интеграла

$$J = \int x^2\ln{x}\space dx$$

$$J = \dfrac{1}{3}\int \ln{x}\space dx^3$$

$$J = \dfrac{1}{3}\left( x^3\ln{x}-\int x^3\space d\ln{x}\right)$$

$$J = \dfrac{1}{3}\left( x^3\ln{x}-\int \dfrac{1}{x}x^3\space dx\right)$$

$$J = \dfrac{1}{3}\left( x^3\ln{x}-\int x^2\space dx\right)$$

$$J = \dfrac{1}{3}\left( x^3\ln{x}-\dfrac{x^3}{3}\right)$$

$$J = \dfrac{x^3\ln{x}}{3} - \dfrac{x^3}{9}$$

Така получаваме

$$\ln{|y|} = \dfrac{Cx^3}{3} + \dfrac{x^3\ln{x}}{12} - \dfrac{x^3}{36} + C_2$$

$$|y| = \mathrm{e}^{\frac{Cx^3}{3} + \frac{x^3\ln{x}}{12} - \frac{x^3}{36} + C_2}$$

$$|y| = \mathrm{e}^{C_2}\mathrm{e}^{\frac{Cx^3}{3} + \frac{x^3\ln{x}}{12} - \frac{x^3}{36}}$$

$$y = \pm\mathrm{e}^{C_2}\mathrm{e}^{\frac{Cx^3}{3} + \frac{x^3\ln{x}}{12} - \frac{x^3}{36}}$$

$$y = C_3\mathrm{e}^{\frac{Cx^3}{3} + \frac{x^3\ln{x}}{12} - \frac{x^3}{36}}$$

\end{document}
