\documentclass{scrartcl}
\usepackage{graphicx}  % required for inserting images

% for cyrillic and Bulgarian language
\usepackage[utf8]{inputenc}
\usepackage[T2A]{fontenc}
\usepackage[bulgarian]{babel}

% math packages
\usepackage{amsmath}
\usepackage{float}
\usepackage{amsfonts}
\usepackage{amssymb}

\usepackage{ragged2e}  % adds FlushLeft

\usepackage[all]{xy}  % fancy arrows
\usepackage{tikz}

\title{Семинар 01}
\author{Ивайло Андреев}
\date{20 февруари 2025}

\begin{document}
\maketitle  % template title

\section{Въведение в курса}

\begin{FlushLeft}
\textcyrillic{Какво са диференциалните уравнения?}
\textcyrillic{Ето примери:}
\end{FlushLeft}

$$y''' x^2 \mathrm{e}^x y^{\sin{x}} + yx = x$$

$$x^2y=y'' \mathrm{e}^x$$

$$y^{(4)} + 2y'' - y = \tan{x}$$

$$y y''' = y'' y' + x^3y$$

\begin{FlushLeft}
Това са уравнения, където търсим неизвестната функцията $y$ и е изпълнено някакво равенство, което (в общия случай) включва първичната функция, нейните производни и нейните аргументи.
\end{FlushLeft}

\begin{FlushLeft}
\textcyrillic{Има два вида диференциални уравнения - обикновени и частни. Обикновените диференциални уравнения (ОДУ) са тези, за които функцията, която търсим, е функция на 1 аргумент. Частните диференциални уравнения (ЧДУ) са тези, за които функцията, която търсим, е функция на 2 или повече аргумента. В по-голямата част от курса ще разглеждаме различни видове ОДУ и към края на курса ще разгледаме някои основни ЧДУ.}
\end{FlushLeft}

\begin{FlushLeft}
\textcyrillic{ОДУ от ред $n$ ще наричаме такова ОДУ, в което имаме $n$-та производна и нямаме производна от по-висок ред.}
\end{FlushLeft}

\begin{FlushLeft}
\textcyrillic{Нека разгледаме най-простите диференциални уравнения. Такива са решавани по ДИС и са въвеждащи.}
\end{FlushLeft}

$$y'(x) = 5x^2$$

$$\displaystyle \int y'(x) \,dx = \int 5x^2\,dx$$

$$\displaystyle \int dy(x) = 5\int x^2\, dx$$

$$y(x) = \dfrac{5}{3} x^3 + C$$

\begin{FlushLeft}
\textcyrillic{Ето и още един пример:}
\end{FlushLeft}

$$y''(x) = x^3 +2x - 8$$

$$\displaystyle \int y''(x)\, dx = \int(x^3 + 2x -8)\, dx$$

$$\displaystyle \int dy'(x) = \int x^3\,dx + \int 2x\,dx + \int (-8)\,dx$$

$$y'(x) = \dfrac{x^4}{4} + 2\dfrac{x^2}{2} -8x + C_1$$

$$\displaystyle \int y'(x)\,dx = \int\left( \dfrac{x^4}{4} + x^2 -8x + C_1\right)$$

$$y(x) = \dfrac{x^5}{20} + \dfrac{x^3}{3} - 4x^2 +C_1x + C_2$$

\begin{FlushLeft}
\textcyrillic{Решението на задачата е именно да намерим явен вид на първичната функция. Разбира се, дясната страна не е нужно да е полином, а произволна функция, която да можем да интегрираме. Можем да забележим, че при ОДУ от ред $n$ получаваме $n$ на брой произволни константи. Това е (интуитивно) защото диференцирането е операция, при която имаме загуба на информация. Диференцирането премахва всички събраеми константи и не можем еднозначно да възвърнем резултата от диференцирането. Ако разглеждаме диференцирането като функция $\phi(f) \rightarrow f'$, която приема функция, и връща функция (именно нейната производна), то $\phi$ НЕ е инекция.}
\end{FlushLeft}

\begin{FlushLeft}
\emph{\textcyrillic{ВАЖНО!}}
\textcyrillic{Това означава, че ОДУ от ред $n$, което има поне едно решение, има безброй много решения или по друг начин казано - 1 решение с точност до $n$ на брой произволни константи.}
\end{FlushLeft}

\section{Интеграли}

\begin{FlushLeft}
\textcyrillic{Както тепърва ще става ясно, диференциалните уравнения се свеждат до интеграли. Затова ще разгледаме интеграли, подходящи за този курс. Показаните интеграли са всички типове интеграли, до които се свеждат диференциални уравнения, решавани на лекции и упражнения и най-вече уравнения, давани на контролни и изпити.}
\end{FlushLeft}

\subsection{}

\begin{align*}
\displaystyle \int \dfrac{x}{x^2+2}\,dx
&= \int \dfrac{1}{x^2+2}\,d\left(\dfrac{x^2}{2}\right) \\
&= \dfrac{1}{2} \int \dfrac{1}{x^2+2}\,d(x^2) \\
&= \dfrac{1}{2} \int \dfrac{1}{x^2+2}\,d(x^2+2) \\
&= \dfrac{1}{2}\ln|x^2+2| + C
\end{align*}

\subsection{}

\begin{align*}
\displaystyle \int \dfrac{\ln x}{x}\,dx
&= \int\dfrac{1}{x} \ln x \,dx \\
&= \int \ln x \, d(\ln x) \\
&= \dfrac{1}{2} \ln^2x + C
\end{align*}

\subsection{}

\begin{align*}
\displaystyle \int x \ln x\,dx
&= \int\ln x \,d\left( \dfrac{x^2}{2} \right) \\
&= \dfrac{1}{2} x^2 \ln x - \int \dfrac{x^2}{2} \, d(\ln x) \\
&= \dfrac{1}{2} x^2 \ln x - \dfrac{1}{2} \int \dfrac{x^2}{x} \, dx \\
&= \dfrac{1}{2} x^2 \ln x - \dfrac{1}{4} x^2 + C
\end{align*}

\subsection{}

\begin{align*}
\displaystyle \int \dfrac{x^2}{1+x^2}\,dx
&= \int \dfrac{x^2+1-1}{x^2+1} \,dx\\
&= \int \left(1 - \dfrac{1}{1+x^2}\right) \,dx\\
&= \int dx - \int \dfrac{1}{1+x^2} \,dx\\
&= x - \arctan{x} + C
\end{align*}

\subsection{}

\begin{align*}
\displaystyle \int \dfrac{1}{\cos{x}}\,dx
&= \int \dfrac{\cos{x}}{\cos^2{x}} \,dx\\
&= \int \dfrac{1}{1-\sin^2{x}} \,d\sin{x}\\
&= \int \dfrac{1}{(1-\sin{x})(1+\sin{x})} \,d\sin{x}\\
&= \int \left(\dfrac{A}{1-\sin{x}} + \dfrac{B}{1+\sin{x}} \right) \,d\sin{x}\\
&= A \int \dfrac{1}{1-\sin{x}}\, d\sin{x} + B \int \dfrac{1}{1+\sin{x}} \, d\sin{x}\\
&= -A \int \dfrac{1}{1-\sin{x}}\, d(1-\sin{x}) + B \int \dfrac{1}{1+\sin{x}} \, d(1+\sin{x})\\
&= -A \ln|1-\sin{x}| + B \ln|1+\sin{x}| + C
\end{align*}

\begin{FlushLeft}
\textcyrillic{Константите $A$ и $B$ намираме по следния начин:}
\end{FlushLeft}

$$\dfrac{1}{(1-u)(1+u)} = \dfrac{A}{1-u} + \dfrac{B}{1+u}$$

$$\dfrac{1}{(1-u)(1+u)} = \dfrac{A + Au + B - Bu}{(1-u)(1+u)}$$

$$1 = (A+B) + u(A-B)$$

$$
\begin{cases}
A-B=0 \\
A+B=1
\end{cases}
$$

$$A = B = \dfrac{1}{2}$$

$$\displaystyle \int \dfrac{1}{\cos{x}}\,dx = \dfrac{1}{2}\ln\left|\dfrac{1+\sin{x}}{1-\sin{x}}\right|+C$$

\subsection{}

\begin{align*}
\displaystyle \int \dfrac{1}{(2x+3)^2}\,dx
&= \dfrac{1}{2}\int \dfrac{1}{(2x+3)^2} \,d(2x+3)\\
&= -\dfrac{1}{2(2x+3)} + C
\end{align*}

\subsection{}

\begin{align*}
\displaystyle \int \dfrac{1}{x^2-4x+5}\,dx
&= \int \dfrac{1}{x^2-4x+4 + 1} \,dx\\
&= \int \dfrac{1}{(x-2)^2 + 1} \,dx\\
&= \int \dfrac{1}{1+(x-2)^2} \,d(x-2)\\
&= \arctan{(x-2)} + C
\end{align*}

\subsection{}

\begin{align*}
\displaystyle \int \dfrac{1}{x^2-4x-5}\,dx
&= \int \dfrac{1}{(x-5)(x+1)} \,dx\\
&= A \int \dfrac{1}{x-5} \,dx + B \int \dfrac{1}{x+1} \,dx\\
&= A\ln|x-5| + B\ln|x+1| + C
\end{align*}

\begin{FlushLeft}
\textcyrillic{Константите $A$ и $B$ намираме по следния начин:}
\end{FlushLeft}

$$\dfrac{1}{(x-5)(x+1)} = \dfrac{A}{x-5} + \dfrac{B}{x+1}$$

$$\dfrac{1}{(x-5)(x+1)} = \dfrac{Ax + A + Bx - 5B}{(x-5)(x+1)}$$

$$1 = (A-5B) + x(A+B)$$

$$
\begin{cases}
A+B=0 \\
A-5B=1
\end{cases}
$$

$$A = \dfrac{1}{6},\quad B = -\dfrac{1}{6}$$

\subsection{}

\begin{align*}
\displaystyle \int \tan{x}\,dx
&= \int \dfrac{\sin{x}}{\cos{x}} \,dx\\
&= -\int \dfrac{1}{\cos{x}} \,d\cos{x}\\
&= -\ln{|\cos{x}|} + C
\end{align*}

\subsection{}

\begin{align*}
\displaystyle \int \tan^2{x}\,dx
&= \int \dfrac{\sin^2{x}}{\cos^2{x}} \,dx\\
&= \int \dfrac{1-\cos^2{x}}{\cos^2{x}} \,dx\\
&= \int \dfrac{1}{\cos^2{x}} \,dx - \int \dfrac{\cos^2{x}}{\cos^2{x}} \,dx\\
&= \tan{x} - x + C
\end{align*}

\subsection{}

\begin{align*}
\displaystyle \int \tan^3{x}\,dx
&= \int \dfrac{\sin^3{x}}{\cos^3{x}} \,dx\\
&= \int \dfrac{\sin^2{x}\sin{x}}{\cos^3{x}} \,dx\\
&= -\int \dfrac{1-\cos^2{x}}{\cos^3{x}} \,d\cos{x}\\
&= -\int \dfrac{1}{\cos^3{x}} \,d\cos{x} + \int \dfrac{1}{\cos{x}} \,d\cos{x}\\
&= \dfrac{1}{2\cos^2{x}} + \ln{|\cos{x}|} + C
\end{align*}

\subsection{}

\begin{align*}
\displaystyle \int \tan^4{x}\,dx
&= \int \dfrac{\sin^4{x}}{\cos^4{x}} \,dx\\
&= \int \dfrac{\sin^2{x}\sin^2{x}}{\cos^4{x}} \,dx\\
&= \int \dfrac{\sin^2{x}(1-\cos^2{x})}{\cos^4{x}} \,dx\\
&= \int \dfrac{\sin^2{x}}{\cos^4{x}} \,dx - \int \dfrac{\sin^2{x}\cos^2{x}}{\cos^4{x}} \,dx\\
&= \int \dfrac{\tan^2{x}}{\cos^2{x}} \,dx - \int \dfrac{\sin^2{x}}{\cos^2{x}} \,dx\\
&= \int \tan^2{x} \,d\tan{x} - \int \tan^2{x} \,dx\\
&= \dfrac{\tan^3{x}}{3} - \tan{x} + x + C
\end{align*}

\hrule

\begin{FlushLeft}
\emph{Следващите наблюдения са изцяло извън рамките на този курс!}

\hfill

Забелязахте ли някакви шаблони при $\tan^{2k}{x}\space dx$?

Далеч не е случайно. Дори е било задача от ДИ (държавен изпит) на специалност Математика, 2008 март.

\hfill

Ето как изглежда формулировката (преправена от мен):

Нека $I(k) = \displaystyle \int \tan^{2k}{x} \space dx$ за $k \in \mathbb{N}\cup \{0\}$

Да се покаже, че $I(k+1) + I(k) = \dfrac{\tan^{2k+1}{x}}{2k+1} + C$

Да се намери общо представяне на $I(k)$, като интегралът се представи като определен в граници $0$ и $\dfrac{\pi}{2}$

\vspace{5pt}

Решение

Нека първо видим първите три начални случая.

$$\displaystyle \int \tan^0{x}\, dx = \int 1 \space dx = x + C$$

$$\displaystyle \int \tan^2{x}\, dx = \dfrac{\tan^1{x}}{1} - x + C$$

$$\displaystyle \int \tan^4{x}\, dx = \dfrac{\tan^3{x}}{3} - \dfrac{\tan^1{x}}{1} + x + C$$

Нека разгледаме $I(k+1)$ и ще го изразим чрез $I(k)$.
\end{FlushLeft}

\begin{align*}
I(k+1)
&= \displaystyle \int \tan^{2(k+1)}{x}\,dx\\
&= \displaystyle \int \tan^{2k+2}{x}\,dx\\
&= \displaystyle \int \tan^2{x}\tan^{2k}{x}\,dx\\
&= \displaystyle \int \left(\dfrac{1}{\cos^2{x}} - 1\right)\tan^{2k}{x}\,dx\\
&= \displaystyle \int \dfrac{1}{\cos^2{x}}\tan^{2k}{x}\,dx - \int \tan^{2k}{x}\,dx\\
&= \displaystyle \int \tan^{2k}{x}\,d\tan{x} - \int \tan^{2k}{x}\,dx\\
&= \dfrac{\tan^{2k+1}{x}}{2k+1} + C - I(k)
\end{align*}

$\implies I(k+1) + I(k) = \dfrac{\tan^{2k+1}{x}}{2k+1} + C$

Оттук не е трудно да се изрази и общия вид на $I(k)$.

\end{document}