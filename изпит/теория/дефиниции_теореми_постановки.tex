\documentclass[11pt]{scrartcl}

\usepackage{fontspec}
\usepackage{amsmath}
\usepackage{graphicx}
\usepackage{enumitem}
\usepackage{listings}
\usepackage{xcolor}
\usepackage{float}
\usepackage{amsfonts}

\setmainfont{DejaVu Serif}

\newcommand{\defcolor}{green}
\newcommand{\withoutproofcolor}{green}
\newcommand{\withproofcolor}{red}
\newcommand{\spaceafter}{\vspace{0.5cm}}

\begin{document}
\begin{flushleft}

\begin{center}
    {\LARGE Уравнения от първи ред}
\end{center}

\spaceafter

$\textit{\color{\defcolor}Дефиниция}$: $f(x, y)$ е
\textbf{липшицова} по $y$ в $\Pi$,
ако $\exists K > 0$,
за което е изпълнено следното неравенство:
$$
|f(x, y_1) - f(x, y_2)| \le K |y_1 - y_2|
$$
където $(x, y_1), (x, y_2) \in \Pi$ са произволни.

\spaceafter

$\textit{\color{\withproofcolor}Лема}$:
Ако $f(x, y), f_y'(x, y) \in C(\Pi) \implies f(x, y)$
е липшицова по $y$ в $\Pi$.

\spaceafter

$\textit{\color{\withoutproofcolor}Теорема}$
(\textbf{Локална теорема за единственост и съществуване}):
Нека $f(x, y)$ е непрекъсната и липшицова по $y$ в
$\Pi := \lbrace(x, y)\in\mathbb{R}:|x - x_0| \le a, |y-y_0| \le b\rbrace$.
Тогава съществува единствено решение на задачата на Коши
$$
\begin{cases}
    y' = f(x, y)
    \\
    y(x_0) = y_0
\end{cases}
$$
в интервала $[x_0-h,x_0+h]$,
където $h = \min \left\lbrace a, \dfrac{b}{M} \right\rbrace$
и $M = \displaystyle \max_{\Pi}|f(x, y)|$.

\spaceafter

$\textit{\color{\defcolor}Дефиниция}$:
Точката $(x_0, y_0) \in D$ е \textbf{обикновена точка}
за уравнението $f(x, y, y') = 0$, ако $f(x_0, y_0, z) = 0$
има краен брой различни и реални решения
$z_1 < z_2 < \dots < z_m$ и
$f'_z(x_0, y_0, z_j) \ne 0 \quad j = 1, 2, \dots, m$

\spaceafter

$\textit{\color{\defcolor}Дефиниция}$:
Точката $(x_0, y_0) \in D$ е \textbf{особена точка}
за уравнението $f(x, y, y') = 0$, ако $f(x_0, y_0, z) = 0$
има реално решение $z_0$ и $f'_z(x_0, y_0, z_0) = 0$

\spaceafter

$\textit{\color{\defcolor}Дефиниция}$:
Решението на уравнението $f(x, y, y') = 0$ е \textbf{особено решение},
ако всички точки от графиката на решението са особени.

\spaceafter

$\textit{Теорема}$ (\textbf{Теорема за редукцията}):
Нека $(x_0, y_0) \in D$ е обикновена точка
за уравнението $f(x, y, y') = 0$.
Тогава в достатъчно малка околност $U \in D$ на точката $(x_0, y_0)$
съществуват функции $f_j(x, y) \quad j=1,2,\dots,m$ такива, че
$f_j, (f_j)'_y \in C(U) \quad f_j(x_0, y_0) = z_j$
и всяко решение на задачата на Коши:
$$
\begin{cases}
    f(x, y, y') = 0
    \\
    f(x_0) = y_0
\end{cases}
$$
е решение на някоя от задачите на Коши:
$$
\begin{cases}
    y' = f_j(x, y)
    \\
    f(x_0) = y_0
\end{cases}
$$

\end{flushleft}
\end{document}